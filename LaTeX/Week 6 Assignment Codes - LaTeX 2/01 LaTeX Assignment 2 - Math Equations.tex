\documentclass[a4paper]{article} %Always requireed
%========================================================================
% preamble to declare kackges used in the document
%========================================================================
%Packages that will be used

%To use a block comment in a document
\usepackage{comment} 
\begin{comment}
	Test a block comment 
\end{comment}
%------------------------------------------------------------------------

\usepackage{fontspec}
\usepackage{enumitem}

\usepackage{amsmath}
%\usepackage{titlesec}

%Either Xelatex or Lualatex is required as a default complier
%To learn more about "fontspec" package: https://ctan.org/pkg/fontspec?lang=en 
%========================================================================
%To declare a document title and an author(s) 
\title{EGCI491 Computer Engineering Seminar\\ \LaTeX Assignment II}
%\author{Firstname Lastname}
%\author{Firstname2 Lastname2}
\author{
	\textbf{Piyachon Rusuwannakul} \\
	\texttt{piyachon.ruu@student.mahidol.edu}
	\and
	\textbf{Ryoichi Kaihatsu} \\
	\texttt{ryoichi.khr@student.mahidol.edu}
	\and
	\textbf{Kiattisak Phisithaporn} \\
	\texttt{kiattisak.pht@student.mahidol.edu}
}

\begin{comment}
	\author{
		LastName1, FirstName1\\
		\texttt{first1.last1@xxxxx.com} 
		%"\textttt" is used to produce text-mode material in typewriter font within a mathematical expression
		
		%To learn more about LaTex command https://www.dickimaw-books.com/latex/novices/html/symbols.html	
		
		\and
		LastName2, FirstName2\\
		\texttt{first2.last2@xxxxx.com}
		\and
		LastName3, FirstName3\\
		\texttt{first3.last3@xxxxx.com}

	}
\end{comment}
%document
\begin{document}
	\maketitle
	\section{\textbf{Quadratic Equation}}
	%Put your text in a body of your document here... 
	A quadratic equation is a second degree polynomial written as $ax^2 + bx + c = 0$
	The generic function of the quadratic equation can be determined as:
	\begin{center}
		\begin{equation}
			f(x) = ax^2 + bx + c
		\end{equation}
		\end{center}
	whose determinant $b^2 - 4ac$ is positive, with x representing an unknown, with a,b, and c representing constants, and with a $\neq$ 0, the quadratic formula is:
	\begin{center}
		\begin{align}
			x = -b \pm \frac{  \sqrt{b^2 - 4ac}}{2a}
			\end{align}
		\end{center}
	where the plus-minus symbol $\pm$ indicates that the quadratic equation has two solutions. When written seperately, they become:
	\begin{align}
		\begin{split}
			x_1 &= -b +  \frac{-b  \sqrt{b^2 - 4ac}}{2a} \hspace*{10pt} and\\
			x_2 &= -b - \frac{-b  \sqrt{b^2 - 4ac}}{2a}
			\end{split}
		\end{align}
\end{document}
%========================================================================